%modif
\documentclass[french,fontsize=22pt]{scrartcl}
%ajout
\usepackage{babel}

%% Recommended to use larger font sizes
%% otherwise, use \Large, \huge liberally :-)

%% Adjust to taste
\usepackage[letterpaper,margin=0.65in,includefoot,bottom=0.3in,footskip=2em]{geometry}

\usepackage[utf8]{inputenc}
\usepackage[T1]{fontenc}
\usepackage[lining,tabular]{fbb}
\usepackage[scaled=.95,type1]{cabin}
\usepackage[varqu,varl]{zi4}
\usepackage[libertine,bigdelims]{newtxmath}

\usepackage[showmarks]{pocketmod}
%\usepackage[]{pocketmod}

\usepackage{graphicx}
\usepackage{enumitem}
\setlist{leftmargin=\parindent,labelindent=*}
\usepackage{hyperref}
\usepackage[hyphenbreaks]{breakurl}

%ajout
\renewcommand{\familydefault}{\sfdefault}
\usepackage{phaistos}
\usepackage{hyphenat}
\hyphenpenalty 10000
\exhyphenpenalty 10000
\usepackage{amsmath}

\title{Pense-bête \\ Tenkara}
\author{Hugues Santerre}
%\date{\texttt{ugolbt@gmail.com}}
\date{}

\setcounter{secnumdepth}{0}

%Compiler en PDFLaTeX, mais éviter XeLaTeX et LuaLaTeX

\begin{document}

\maketitle
\setcounter{page}{0}
\thispagestyle{empty}

\vspace*{\stretch{1}}
{\centering
	\includegraphics[width=0.4\textwidth]{img/truite1}
	\par
}
\vspace*{\stretch{1}}

\clearpage


\section{Salmonidés}

\textbf{Ombles, ombres, ouananiches, truites}

\begin{itemize}[label={}]
	\item \PHtunny~ mouches noyées et sèches; nymphes
	\item \PHtunny~ Muddler Minnow (peut être doublé d'une sèche)
	\item \PHtunny~ Wolly Bugger lent (peut être doublé d'une sèche)
	\item \PHtunny~ train à 3 mouches (1 sèche, 2 noyées ou autres leurres)
	\item \PHtunny~ streamer
\end{itemize}

\textbf{Gros salmonidés: saumons, truite \textit{steelhead}, etc.}

\begin{itemize}[label={}]
	\item \PHtunny~ grossir les leurres des petits salmonidés
\end{itemize}

\vspace*{\stretch{1}}
\Large
\textbf{Présentations}
\normalsize

\begin{itemize}[label={}]
	\item \PHtunny~ inanimées: neutre (avec le courant), avec pause (contre le courant), neutre avec pause, courte (touche rapide), en profondeur (couler dans la colonne d'eau; grosse mouche ou mouche lestée)
	\item \PHtunny~ animées: avec vibrations (tapements), tirée ou trainées, tirées perpendiculairement, pulsée (avec une canne $5:5$ ou $6:4$)
\end{itemize}

\clearpage


\section{Carnassiers}

\textbf{Achigans}

\begin{itemize}[label={}]
	\item \PHtunny~ Clouser Minnow
	\item \PHtunny~ popper
	\item \PHtunny~ araignée (dans les herbiers)
\end{itemize}

\textbf{Brochets et maskinongés}

\begin{itemize}[label={}]
	\item \PHtunny~ popper
	\item \PHtunny~ streamer
\end{itemize}

\textbf{Crapets, mariganes, perchaudes, perches soleil}

\begin{itemize}[label={}]
	\item \PHtunny~ Muddler Minnow (peut être doublé d'une sèche)
	\item \PHtunny~ Wolly Bugger lent (peut être doublé d'une sèche)
	\item \PHtunny~ streamer
\end{itemize}

\vspace*{\stretch{1}}
\Large
\textbf{Autres?}
\normalsize

\clearpage


\section{Autres Poissons}

\textbf{Aloses}

\begin{itemize}[label={}]
	\item \PHtunny~ streamer
\end{itemize}

\textbf{Bar rayé}

\begin{itemize}[label={}]
	\item \PHtunny~ Muddler Minnow
	\item \PHtunny~ Clouser Minnow
	\item \PHtunny~ Wolly Bugger (lent)
	\item \PHtunny~ streamer 
\end{itemize}

\vspace*{\stretch{1}}
\Large
\textbf{Autres?}
\normalsize

\clearpage


\section{Lignes}

\begin{itemize}[label={}]
	\item \PHtunny~ \textbf{ligne tressée}: idéal pour les présentations inanimées; flotte
	\item \PHtunny~ \textbf{ligne parallèle}: idéal pour les présentations animées: coule
\end{itemize}

\textbf{Diamètres de la ligne parallèle}

\begin{itemize}[label={}]
	\item \PHtunny~ \#2 et \#2.5: canne $5{:}5$ et $6{:}4$, sans vent, peu profond, hameçon 14 et -
	\item \PHtunny~ \#3: idem, mais plus venteux et/ou plus profond, hameçon 14
	\item \PHtunny~ \#3.5: toute canne, hameçon 12 (\textbf{standard})
	\item \PHtunny~ \#4: idem, canne plus longue et/ou plus venteux et/ou hameçon 10
	\item \PHtunny~ \#4.5: canne $7{:}3$ et $8{:}2$, canne plus longue et/ou plus venteux et/ou hameçon 10 et +
\end{itemize}

\vspace*{\stretch{1}}
\Large
\textbf{Autres?}
\normalsize

\clearpage


\section{Tailles des bas de ligne}

\textbf{À la noyée et autres leurres noyés}

\begin{itemize}[label={}]
	\item \PHtunny~ ruisseau, peu profond: 30cm-1pi
	\item \PHtunny~ petite rivière: 90cm-3pi
	\item \PHtunny~ moyenne à grande rivière, profond: 120cm-4pi
	\item \PHtunny~ courant: + 15-30cm-6-12po (++ en eaux très vives)
	\item \PHtunny~ venteux: - 15-30cm-6-12po (aide, mais dépend plus de la ligne)
\end{itemize}

\textbf{Autres techniques}

\begin{itemize}[label={}]
	\item \PHtunny~ en sèche: le bas de ligne doit flotter (avec du flottant): 30cm-1pi ou +
	\item \PHtunny~ à la nymphe: la ligne doit couler à une certaine profondeur: 30cm-1pi permet de mesurer la longueur entre un jalon sur la ligne et la nymphe 
\end{itemize}

\vspace*{\stretch{1}}
\Large
\textbf{Autres?}
\normalsize

\clearpage


\section{Tensions des bas de ligne, hameçons}

\begin{itemize}[label={}]
	\item \PHtunny~ poisson-appât: 1.0kg-2.2lb, 22-18 
	\item \PHtunny~ petit poisson: 1.4kg-3.1lb, 18-14
	\item \PHtunny~ 25-30cm: 1.8kg-4.0lb, 16-12
	\item \PHtunny~ 30-40cm: 2.4kg-5.3lb, 14-10 (\textbf{standard})
	\item \PHtunny~ 40-50cm*, 3.0kg-6.6lb, 12-8
	\item \PHtunny~ petit achigan*, 4.2kg-9.2lb, 10-6
	\item \PHtunny~ gros salmonidés et carnassier*, 5.1kg-11.2lb, 8-4
	\item \PHtunny~ gros saumon et carnassier*, 6.4kg-14.1lb, 6-2
	\item \PHtunny~ poisson d'eau salée*, 7.3kg-16.0lb, hameçon 4-1/0
	\item $\ast$ \textbf{Attention!} Avec un canne $7{:}3$ ou $8{:}2$ robuste
\end{itemize}

\vspace*{\stretch{1}}
\Large
\textbf{Températures}
\normalsize

\begin{itemize}[label={}]
	\item \PHtunny~ eau froide (10-15°C): aloses, bar rayé et salmonidés (ombles, ombres, saumons, truites)
	\item \PHtunny~ eau tempérée (15-21°C): carnassiers (achigans à petites bouche, brochets et maskinongés)
	\item \PHtunny~ eau chaude (21-27°C): carnassiers (achigans à grande bouche, crapets, mariganes, perchaudes, perches soleil)
\end{itemize}

\clearpage


\section{Silhouettes (mais les tailles varient)}

\PHtunny~ centrarchidés:

\hspace{\parindent}\PHtunny~ crapets, mariganes, perches soleil, perchaudes

{\centering
	\includegraphics[width=0.35\textwidth]{img/centrarchides}
	\par
}

\hspace{\parindent}\PHtunny~ achigans

{\centering
	\includegraphics[width=0.35\textwidth]{img/centrarchidesb}
	\par
}

\PHtunny~ ésocidés: brochets et masquinongés

{\centering
	\includegraphics[width=0.35\textwidth]{img/esocides}
	\par
}

\PHtunny~ salmonidés:

\hspace{\parindent}\PHtunny~ ombles cristivomer, dolly varden, etc.

{\centering
	\includegraphics[width=0.35\textwidth]{img/salmonides\_ombles}
	\par
}

\hspace{\parindent}\PHtunny~ saumons, ouananiches

{\centering
	\includegraphics[width=0.35\textwidth]{img/salmonides\_saumons}
	\par
}

\hspace{\parindent}\PHtunny~ truites

{\centering
	\includegraphics[width=0.35\textwidth]{img/salmonides\_truites}
	\par
}

\hspace{\parindent}\PHtunny~ ombres, corégones, cisco, etc.

{\centering
	\includegraphics[width=0.35\textwidth]{img/salmonides\_ombres}
	\par
}

\PHtunny~ clupéidés: alose (similaire aux sardines et harengs)

{\centering
	\includegraphics[width=0.15\textwidth]{img/clupeides\_alose}
	\par
}

\PHtunny~ moronidés: bar rayé

{\centering
	\includegraphics[width=0.15\textwidth]{img/moronides\_bar}
	\par
}

\end{document}
